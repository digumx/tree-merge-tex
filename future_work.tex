\section{Conclusion and Future Work}

\dmcmt{Regd comment from Review 3, there seems to not be much repetation between
this and Intro right?}

This paper puts forth a framework to combine syntactic and semantic approaches
for DNN abstraction. While this opens several directions for future work, an
immediate question that can be asked is about achieving ``optimal refinement''.
In the abstract network, it would be ideal to not have neurons that can be
merged back without introducing spurious counterexamples. This implies that we
should have added the minimum number of neurons necessary to prevent spurious
counterexamples. This seems to depend on the direction in which one starts
refining. Of the multiple refinement paths that are possible, is there one that
is guaranteed to do a \emph{minimal} refinement? \dmcmt{Maybe re-written?}

Exploring alternative measures for the semantic closeness factor $\cls$ is
another intriguing prospect. Note that our framework allows one to seamlessly
experiment with any $\cls$. Our current $\cls$ does capture optimality with
respect to the I/O behavior of the neurons, but we are not able to guarantee
minimization of the over-approximation at the output layer. It would help to
identify better metrics for the semantic closeness factor that minimize
over-approximation at the output neuron. \dmcmt{If we talk about characteristic
merge as other option, maybe we can add this here?}

In our experiments, we noticed that the time needed to verify a specific
property of a network is not solely determined by the network's size. It is
possible that a larger network can be verified quickly, while a smaller one may
take longer or even fail to be verified altogether. Obtaining a more accurate
measure of the effort required to verify a network would provide a better
optimization target for abstraction used in the context of verification.



