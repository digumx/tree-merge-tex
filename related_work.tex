\section{Related Work}

\sncmt{Remove first two paras here} \dmcmt{Agreed}

In response to the increasing use of neural networks in safety-critical settings, 
various techniques have been developed to analyze their behavior, including 
verification, explainability, etc. 

Our work closely relates to the syntactic structural abstraction techniques that
propose to reduce the network size by merging similar
neurons \cite{cegar-nn,cegarette,cleverest-nn}, and employs a
counterexample-guided refinement. They have used
syntactic constraints to achieve concrete soundness guarantees. But unlike our
approach, their work does not take into account the global \textit{semantic}
behavior of the network, thus producing potentially suboptimal abstractions.
At the same time, the approach proposed in \cite{deep-abstract} performs a
semantic analysis to decrease the network size, but it leads to an
\emph{approximation} instead of an abstraction. The approximation allows
lifting certain bound propagation based proofs, but not the concrete gurantee
that one gets from~\cite{cegar-nn}. Also, in \cite{lin-comb-abs-jan} the
authors have used linear combinations of neurons to compress the networks, but
their method only provides guarantees on a finite dataset. In comparison, we
provide concrete soundness guarantees that allows us to lift any proofs from
the abstract network back to the original network.

\dmcmt{Alt section starts from here}

In general, methodologies for verifying neural networks generally fall into two
main categories: sound and complete methods
\cite{reluplex,formal-ver-piece-wise,comp-reachability-analysis,comp-milp,comp-out-range,comp-max-resilience,marabou,comp-safety-ver-dnn,beta-crown,alpha-crown-bab-fnc,gcp-crown},
and sound and incomplete methods
\cite{deeppoly,crown,incomp-dual-approach,incomp-abs-inp,incomp-robustness-certi,incomp-boost-robustness}.
Sound and complete methods aim to explore the entire state space to verify the
properties of neural networks.
In contrast, sound and incomplete methods employ an overapproximation
of the state space, sacrificing completeness for 
scalability and efficiency.

An instance of a sound and complete methodology is Reluplex, which extends the 
simplex algorithm \cite{simplex} to 
handle ReLU constraints. Initially, it focuses on finding an assignment that 
satisfies the linear constraints, subsequently incorporating non-linear constraints 
to validate their satisfaction. In \cite{formal-ver-piece-wise}
the authors introduce triangular over-approximation, infer neuron phase fixtures,
learn conflict clauses and safe neuron phase fixtures to aid in pruning the search 
space, which is similar to classical SMT solving approaches. Another complete
technique is \neuralsat, which performs exhaustive theory propagation and
conflict clause learning similar to DPLL(T) used in classical SMT solving.
However, these methods often encounter scalability issues due to their
exhaustive exploration of the entire state space. 

On the other hand, alternative techniques like \cite{crown,deeppoly}, which
propagates linear upper and lower bound constraints, exhibit better
scalability at the cost of overapproximation. \cite{alpha-crown-bab-fnc} 
optimizes the bounds of \cite{crown} 
using gradient descent. 
\cite{beta-crown} incorporates ReLU split
constraints into the CROWN bound propagation process, allowing integration
into the branch and bound (BaB) framework
\cite{bab-fw,bab-piecewise-nn,bab-lagrangian-decomp}. 
This combined implementation makes the \abcrown framework sound and complete.

DNN compression has been explored \cite{dnn-compression} in general to obtain a
network of a smaller size, however, as opposed to abstraction, they do not
provide any guarantees connecting the original network to the smaller compressed
network. 
