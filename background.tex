\section{Background}
\subsection{Notation}

We restrict ourselves to fully-connected, feed-forward neural networks with
\relu activation function. 
Neurons in our network are denoted as $\nr{i}{l}$, where $i$ signifies the 
neuron number in layer $l$.  % Used
The function taking a given input $\vct{x}$ to the value of $\nr{i}{l}$ is
represented by $\nrf{i}{l}(\vct{x})$, % Used 
The function $\ob{i}{l}$ takes a list of input vectors
$[\vct{x_1} {\cdots} \vct{x_N}]$ and returns a vector with
$[\nrf{i}{l}(\vct{x_1}) {\cdots} \nrf{i}{l}(\vct{x_N})]$ for a particular
neuron $\nr{i}{l}$.  %Used

\subsection{Formal Analysis of Neural Networks }
\label{s:form-an}

Several techniques and methods have been studied to improve the reliability and
trustworthiness of \dnn deployed in safety-critical settings via formal
analysis. This includes verifying \dnn with respect to a given
safety property \cite{reluplex, cegar-nn, deeppoly, cegarette, cleverest-nn,
conv-abs-gk, deep-abstract, lin-comb-abs-jan}, providing formal explanations of
the behavior of \dnn \cite{minimal-image-fxai, overview-fxai}, and defending
against backdoor attacks \cite{backdoor-verification}.
 To provide the formal guarantees behind the
analyses performed, all of these techniques rely on making \textit{neural
network queries}. 

These neural network queries are of the form $(P, \mcnc, Q)$, and asks: if
for all inputs $\vct{x}$ to $\mcnc$ for which the formula $P$ holds, 
the formula $Q$ also hold on the output $\mcnc(\vct{x})$. While there are
several
tools that can handle such queries, like \marabou and \abcrown, scalability
remains an issue, and so reducing the size of \cnc is desirable.

\subsection{Semantic Compressions and Abstractions with Empirical Guarantees}
\label{s:emp-abs}

Several techniques utilize semantic information, typically extracted via
simulation of the \dnn, to obtain a smaller \abs. 
Neural network compression techniques \cite{dnn-compression}, produce small
\abs, but the
behavior of \abs in connection to \cnc is only characterized empirically.
Similarly, some semantic abstraction techniques \cite{lin-comb-abs-jan} provide
bi-simulation guarantees bounding the difference in the behavior of \abs and
\cnc on a finite set of input points, typically a subset of the training
dataset. Since these techniques characterize the behavior of \abs only on a
finite set of
input points, the trust obtained on the connection between \cnc and \abs is only
of a empirical nature.
Other techniques like \cite{deep-abstract} use bi-simulation to lift
interval bound propagation performed on \abs to get sound bounds on \cnc. While
this does provide a sound proof, interval bounds are typically not strong enough
to prove many interesting and practically relevant neural network queries.

\subsection{Concrete Guarantees on Neural Network Abstractions}
\label{s:conc-abs}

The notion of providing concrete guarantees on the behavior of \abs relative to
\cnc has been formalized in \cite{cegar-nn}. In particular, guarantees of the
form $\forall x, \mabs(x) \geq \mcnc(x)$ are useful as a general notion of concrete
guarantees in many settings. Two such settings are as follows: 

Firstly, since any general neural network query can be converted to a query of
the form $(P, \mcnc, y < c)$ for some $c$ \cite{cegar-nn, reluplex,
lipschitz-reach}, such a guarantee is
useful for dispatching the query on a smaller network \cite{cegar-nn, cegarette,
cleverest-nn}. This immediately makes an \abs with these guarantees useful for
accelerating several formal analysis techniques (Section \ref{s:form-an}).

Furthermore, such guarantees are also useful for safe compression of \dnn.
Consider the case of medical diagnosis \cite{b1} or aircraft
collisions\cite{acasxu}, where
for safety reasons, a classifier should be biased towards not producing false
negatives. Guarantees such as above can formally ensure that the compressed \abs
never produces any more false negatives than \cnc (Section
\ref{s:exp-mnist-comp}).

Therefore, in this work we focus on developing a framework that produces
abstract networks with this guarantee.

\subsection{Quality of Abstractions and \gencex}
\label{s:qual}

A generally useful notion with respect to abstraction is \textit{quality},
which we define as the number of \textit{\gencex} that witness a difference in
the relevant behavior of \abs and \cnc. In the context of using \abs to
accelerate formal analysis of \dnn, these \gencex may simply be
\textit{spurious counterexamples} \cite{cegar-nn, cleverest-nn} to a query
involving \abs that are not counterexamples for the same query involving \cnc.
This notion may be generalized to other uses of abstractions as well. For
instance, for safe compressions, they may be inputs from a dataset which
are falsely classified as positive by \abs (Section \ref{s:exp-mnist-comp}).

\subsection{Syntactic Neural Network Splitting and Merging}
\label{s:nn-sam}

In order to ensure the soundness guarantees when transforming \cnc into \abs, 
instead of adopting the four-way split approach proposed in \cite{cegar-nn},
we opt for a two-way split method \cite{chauhan2022efficiently,liu2022abstraction,
10.1145/3644387}. This involves partitioning neurons in \cnc into duplicate
copies labeled with either {\inc or \dec}, ensuring that any increase 
(decrease) in the value of a neuron labeled \inc (\dec) only results in 
an increase in the output value. This can be done because it is not
necessary to perform \posc-\negc splitting to soundly obtain an \inc-\dec
split, as seen in \cite{chauhan2022efficiently,liu2022abstraction,
10.1145/3644387}. Then, following \cite{cegar-nn}, a sound 
abstraction can be obtained by \textit{merging} all similarly 
labeled neurons as follows: if the neurons have
the label \inc (\dec), replace incoming edges from the same
previous layer neuron with a single edge with the maximum (minimum) of the
original edge weights. Outgoing edges to the same next layer neuron are replaced
with a single edge with the sum of the weights for both the \inc and \dec case.

%\kmcmt{Let us use these references for the 2-class classification:~\cite{chauhan2022efficiently,liu2022abstraction,10.1145/3644387}. Also, I think it might be better to state the 2-class directly as a modification. Something like, ``Instead of the four-way split, we consider only a two-way split of neurons as \emph{inc} and \emph{dec}.'' And then let us preserve the term ``re-merging'' for merging an increment and its corresponding decrement neuron back into one. I think the following para, and possibly relevant parts of the following sections, will have to be changed accordingly.}

\begin{figure}[htbp]
    \centering
    \begin{tikzpicture}[scale=0.5] % Adjust the scale factor as needed
      % Your TikZ code goes here
      \draw (5,-5) circle (1cm);
      \node at (5,-5) {$n_{(1,3,+)}$};
      \draw (5,-2) circle (1cm);
      \node at (5,-2) {$n_{(1,2,+)}$};
      \draw (0,2) circle (1cm);
      \node at (0,2) {$n_{(0,0)}$};
      \node at (-2,2) {[-1, 1]};
      \draw (0,-2) circle (1cm);
      \node at (-2,-2) {[-1, 1]};
      \node at (0,-2) {$n_{(0,1)}$};
      \draw (10,0) circle (1cm);
      \node at (10,0) {$n_{(2,0)}$};
      \node at (12,0) {$< 6.1$};
      \draw (5, 5) circle (1cm);
      \node at (5,5) {$n_{(1,0,+)}$};
      \draw (5, 2) circle (1cm);
      \node at (5,2) {$n_{(1,1,+)}$};

    \draw (1, -2) -- (4, -5);
    \node at (3.25,-4.75) {0.5};
    \draw (1, 2) -- (4, -5);
    \node at (4,-4) {1};
    \draw (1, -2) -- (4, -2);
    \node at (3.45,-2.5) {0.55};
    \draw (1, 2) -- (4, -2);
    \node at (4.25,-0.75) {0.95};
    \draw (1, -2) -- (4, 2);
    \node at (4.25, 0.75) {0.95};
    \draw (1, -2) -- (4, 5);
    \node at (4, 4) {1};
    \draw[solid, postaction={decorate, decoration={text along path,
          text={0.55}, text align=right, raise=1mm}}] (1, 2) -- (4, 2);
    \draw[solid, postaction={decorate, decoration={text along path,
          text={0.5}, text align=right, raise=1mm}}] (1, 2) -- (4, 5);

    \draw[solid, postaction={decorate, decoration={text along path,
        text={1}, text align=center, raise=1mm}}] (6,-5) -- (9, 0);
    \draw[solid, postaction={decorate, decoration={text along path,
        text={1}, text align=center, raise=1mm}}] (6,-2) -- (9, 0);
    \draw[solid, postaction={decorate, decoration={text along path,
        text={1}, text align=center, raise=1mm}}] (6,2) -- (9, 0);
    \draw[solid, postaction={decorate, decoration={text along path,
        text={1}, text align=center, raise=1mm}}]  (6, 5) -- (9, 0);
      
    \end{tikzpicture}
    \caption{Original Network and Property}
    \label{fig:Original_Net_Property}
  \end{figure}
  

\begin{figure}[htbp]
    \centering
    \begin{tikzpicture}[scale=0.5] % Adjust the scale factor as needed
      % Your TikZ code goes here
      \draw[line width=1pt, fill=orange!30] (5,0) circle (1cm);
      \node at (5,0) {$a_1$};
      \node [text=red] at (5,1.5) {\textbf{1.55}};
      \draw[line width=1pt, fill=blue!30] (0,2) circle (1cm);
      \node at (0,2) {$n_{(0,0)}$};
      \node [text=red] at (-2,2) {\textbf{0.55}};
      \draw[line width=1pt, fill=blue!30] (0,-2) circle (1cm);
      \node [text=red] at (-1.5,-2) {\textbf{1}};
      \node at (0,-2) {$n_{(1,0)}$};
      \draw[line width=1pt, fill=green!30] (10,0) circle (1cm);
      \node at (10,0) {$n_{(0,2)}$};
      \node [text=red] at (12,0) {\textbf{6.2}};
    \draw[myarrow] (1, -2) to (4, 0);
    \node at (2.5,-1.25) {1};
    \draw[myarrow] (1, 2) to (4, 0);
    \node at (2.5,1.25) {1};


    \draw[->,  >=angle 45, thick, postaction={decorate, decoration={text along path,
        text={4}, text align=center, raise=1mm}}] (6,0) -- (9, 0);
    % \draw[solid, postaction={decorate, decoration={text along path,
    %     text={1}, text align=center, raise=1mm}}] (6,-2) -- (9, 0);
    % \draw[solid, postaction={decorate, decoration={text along path,
    %     text={1}, text align=center, raise=1mm}}] (6,2) -- (9, 0);
    % \draw[solid, postaction={decorate, decoration={text along path,
    %     text={1}, text align=center, raise=1mm}}]  (6, 5) -- (9, 0);
      
    \end{tikzpicture}
    \caption{Fully Abstracted Network}
    \label{fig:full_abstract_net}
  \end{figure}
  


For example, consider the network and property in Figure
\ref{fig:Original_Net_Property}. For this example, all the neurons in the output
and middle layer get classified as \inc. Thus, they are all merged together,
leading to the network in Figure \ref{fig:full_abstract_net}.

Note that this process only considers the syntactic structure of the network, no
semantic information is used.

\subsection{ Syntactic Refinement }

The fully merged network obtained in Section \ref{s:nn-sam} may not have
sufficient quality (Section \ref{s:qual}) to be useful. For instance, when doing
formal analysis, there may be too many spurious counterexamples. In such
situations, a common approach to obtain a better quality \abs is to perform
refinement steps based on a \gencex $\vct{\beta}$ \cite{cegar-nn,
cegarette, cleverest-nn}. In
existing techniques, this is typically done by restoring a single neuron coming
from \cnc that had been merged with other neurons in \abs. The neuron chosen is
typically one whose contribution to $\vct{\beta}$ is estimated to be the
highest.

These techniques, however, do not consider any semantic behavior to guide their
refinement. As such, the refinement process tends to produce a large number of
restored neurons that are not merged with any other neurons. A proliferation of
these \emph{singleton} neurons leads to \abs having a larger size. At the same time, 
a large group of neurons remains merged in \abs, which affects the quality of \abs. 

We can see this in our example. Say the fully merged network in Figure
\ref{fig:full_abstract_net} we get a $\vct{\beta}$ given by the values in \textbf{bold}.
Then, in the next refinement step, the neuron $\nr{3}{1}$ gets restored, giving
us the network in Figure \ref{fig:refine_step_1}. Again, the $\vct{\beta}$
obtained is shown in \textbf{bold}, and the next refinement step restores $\nr{0}{1}$
leading to Figure \ref{fig:refine_step_2}. We see that in the resultant network,
two semantically dis-similar neurons $\nr{1}{1}$ and $\nr{2}{1}$ remain merged,
while the merges between the similar pairs of neurons $\nr{3}{1}$, $\nr{2}{1}$
and $\nr{1}{1}$, $\nr{0}{1}$ have been un-done. Indeed, the network in Figure
\ref{fig:refine_step_2} is not strong enough, as seen by the values in \textbf{bold}, and
we end up refining all the way to the original network.

\begin{figure}[htbp]
    \centering
    \begin{tikzpicture}[scale=0.5] % Adjust the scale factor as needed
      % Your TikZ code goes here
      \draw[line width=1pt, fill=orange!30] (5,2) circle (1cm);
      \node at (5,2) {$a_2$};
      \node [text=red] at (5,0.75) {1.7};
      \draw[line width=1pt, fill=pink!30] (5,-2) circle (1cm);
      \node at (5,-2) {$n_{(3,1)}$};
      \node [text=red] at (5,-3.5) {1.375};
      \draw[line width=1pt, fill=blue!30] (0,2) circle (1cm);
      \node at (0,2) {$n_{(0,0)}$};
      \node [text=red] at (-2,2) {1};
      \draw[line width=1pt, fill=blue!30] (0,-2) circle (1cm);
      \node [text=red] at (-1.75,-2) {0.75};
      \node at (0,-2) {$n_{(1,0)}$};
      \draw[line width=1pt, fill=green!30] (10,0) circle (1cm);
      \node at (10,0) {$n_{(0,2)}$};
      \node [text=red] at (12,0) {$6.475$};
    \draw[->, >= angle 45, thick, postaction={decorate, decoration={text along path,
    text={0.5}, text align=center, raise=1mm}}](1, -2) -- (4, -2);
   
    \draw[myarrow] (1, -2) -- (4, 2);
    \node at (3.75,1.25) {1};
  
    \draw[myarrow] (1, 2) -- (4, -2);
    \node at (3.75,-1.25) {1};
  
    \draw[->, >= angle 45, thick, postaction={decorate, decoration={text along path,
    text={0.95}, text align=center, raise=1mm}}] (1, 2) -- (4, 2);
   


    \draw[->, >= angle 45, thick, postaction={decorate, decoration={text along path,
    text={1}, text align=center, raise=1mm}}] (6,-2) -- (9, 0);
    \draw[->, >= angle 45, thick, postaction={decorate, decoration={text along path,
    text={3}, text align=center, raise=1mm}}] (6, 2) -- (9, 0);

    \end{tikzpicture}
    \caption{Refine Step 1: Culprit Neuron is 3}
    \label{fig:refine_step_1}
  \end{figure}
  

\begin{figure}[htbp]
    \centering
    \begin{tikzpicture}[scale=0.5] % Adjust the scale factor as needed
      % Your TikZ code goes here
      \draw (0,2) circle (1cm);
      \node at (0,2) {$n_{(0,0)}$};
      \node [text=red] at (-2,2) {1};
      \draw (0,-2) circle (1cm);
      \node [text=red] at (-2,-2) {1};
      \node at (0,-2) {$n_{(1,0)}$};
      \draw (5,4) circle (1cm);
      \node at (5,4) {$n_{(0,1)}$};
      \node [text=red] at (5,2.5) {1.5};
      \draw (5,-4) circle (1cm);
      \node at (5,-4) {$n_{(3,1)}$};
      \node [text=red] at (5,-5.5) {1.5};
      \draw (5,0) circle (1cm);
      \node at (5,0) {$a_3$};
      \node [text=red] at (5,-1.5) {1.9};
      \draw (10,0) circle (1cm);
      \node at (10,0) {$n_{(0,2)}$};
      \node [text=red] at (12,0) {$6.8$};
    \draw[->, >= angle 45, postaction={decorate, decoration={text along path, 
    text={0.5}, text align=center, raise=1mm}}](1, 2) -- (4, 4);
    \draw[myarrow](1, -2) -- (4, 4);
    \node at (3.75,3) {1};
    \draw[myarrow](1, 2) -- (4, 0);
    \node at (3.75,1) {0.95};
    \draw[myarrow](1, -2) -- (4, 0);
    \node at (3.75,-1) {0.95};
    \draw[myarrow](1, 2) -- (4, -4);
    \node at (3.75,-3) {1};
    \draw[->, >= angle 45, postaction={decorate, decoration={text along path, 
    text={0.5}, text align=center, raise=1mm}}](1, -2) -- (4, -4);
    \draw[->, >= angle 45, postaction={decorate, decoration={text along path,
    text={1}, text align=center, raise=1mm}}] (6,-4) -- (9, 0);
    \draw[-> , >= angle 45 , postaction={decorate, decoration={text along path,
    text={2}, text align=center, raise=1mm}}] (6,0) -- (9, 0);
    \draw[->, >= angle 45, postaction={decorate, decoration={text along path,
    text={1}, text align=center, raise=1mm}}] (6,4) -- (9, 0);
   
    % \draw (1, -2) -- (4, 2);
    % \node at (3.75,1.25) {1};
  
    % \draw (1, 2) -- (4, -2);
    % \node at (3.75,-1.25) {1};
  
    % \draw[solid, postaction={decorate, decoration={text along path,
    % text={0.95}, text align=center, raise=1mm}}] (1, 2) -- (4, 2);
   


    % \draw[solid, postaction={decorate, decoration={text along path,
    % text={1}, text align=center, raise=1mm}}] (6,-2) -- (9, 0);
    % \draw[solid, postaction={decorate, decoration={text along path,
    % text={3}, text align=center, raise=1mm}}] (6, 2) -- (9, 0);

    \end{tikzpicture}
    \caption{Refine Step 2: Culprit Neuron is 0}
    \label{fig:refine_step_2}
  \end{figure}
  

