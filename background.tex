\section{Background}
\subsection{Notation}

We restrict ourselves to fully-connected, feed-forward neural networks with
\relu activation function. 
Neurons in our network are denoted as $\nr{i}{l}$, where $i$ signifies the 
neuron number in layer $l$.  % Used
The function taking a given input $\vct{x}$ to the value of $\nr{i}{l}$ is
represented by $\nrf{i}{l}(\vct{x})$, % Used 
The function $\ob{i}{l}$ takes a list of input vectors
$[\vct{x_1} {\cdots} \vct{x_N}]$ and returns a vector with
$[\nrf{i}{l}(\vct{x_1}) {\cdots} \nrf{i}{l}(\vct{x_N})]$ for a particular
neuron $\nr{i}{l}$.  %Used

\subsection{Formal Analysis of Neural Networks }
\label{s:form-an}

Several techniques and methods have been studied to improve the reliability and
trustworthiness of \dnn deployed in safety-critical settings via formal
analysis. This includes verifying \dnn with respect to a given
safety property
\cite{reluplex,cegar-nn,deeppoly,cegarette,cleverest-nn,conv-abs-gk,deep-abstract,lin-comb-abs-jan},
providing formal explanations of
the behavior of \dnn \cite{minimal-image-fxai,overview-fxai}, and defending
against backdoor attacks \cite{backdoor-verification}.
 To provide the formal guarantees behind the
analyses performed, all of these techniques rely on making \textit{neural
network queries}. 

These neural network queries are of the form $(P, \mcnc, Q)$ and ask: if
for all inputs $\vct{x}$ to $\mcnc$ for which the formula $P$ holds, 
the formula $Q$ also holds for the output $\mcnc(\vct{x})$. While there are
several
tools that can handle such queries, like \marabou, \abcrown and \neuralsat,
scalability remains an issue, and so reducing the size of \cnc is desirable.

\subsection{Semantic Compressions and Abstractions with Empirical Guarantees}
\label{s:emp-abs}

Several techniques utilize semantic information, typically extracted via
simulation of the \dnn, to obtain a smaller \abs. 
Neural network compression techniques \cite{dnn-compression}, produce small
\abs, but the
behavior of \abs in connection to \cnc is only characterized empirically.
Similarly, some semantic abstraction techniques \cite{lin-comb-abs-jan} provide
bi-simulation guarantees bounding the difference in the behavior of \abs and
\cnc on a finite set of input points, typically a subset of the training
dataset. Since these techniques characterize the behavior of \abs only on a
finite set of
input points, the trust obtained on the connection between \cnc and \abs is only
of an empirical nature.
Other techniques, like \cite{deep-abstract}, use bi-simulation to lift
interval bound propagation performed on \abs to get sound bounds on \cnc. While
this does provide a sound proof, interval bounds are typically not strong enough
to prove many interesting and practically relevant neural network queries.

\subsection{Strong Formal Connections between \cnc and \abs}
\label{s:conc-abs}

Structural abstraction techniques, including \cite{cegar-nn}, provide the
following strong formal connection between the behavior of \cnc and \abs:
$\forall x, \mabs(x) \geq \mcnc(x)$. 
Any general neural network query can be converted to a query of
the form \linebreak $(P, \mcnc, y < c)$ for some $c$
by encoding the postcondition as extra layers in \cnc 
\cite{cegar-nn,reluplex,lipschitz-reach}.
Therefore, this notion of formal connection allows one to abstract the larger
\cnc to a smaller \abs, dispatch the
easier query $(P, \mabs, y < c)$ using a solver call, and argue that the
original query holds \cite{cegar-nn,cegarette,cleverest-nn}. 
This immediately makes such an \abs useful
for accelerating several formal analysis techniques (Section \ref{s:form-an}).
Therefore, in this work, we focus on developing a framework that produces
abstract networks that maintain this formal connection.

\subsection{Syntactic Neural Network Splitting and Merging}
\label{s:nn-sam}

In order to ensure soundness guarantees when transforming \cnc into \abs, 
we adopt a modification of the four-way split approach proposed in
\cite{cegar-nn}.
In this approach, a two-way split is used instead
\cite{chauhan2022efficiently,liu2022abstraction,10.1145/3644387}, 
partitioning neurons in \cnc into duplicate
copies labeled with either {\inc or \dec}. 
This partitioning is done in a way that 
ensures that any increase 
(decrease) in the value of a neuron labeled \inc (\dec) only results in 
an increase in the output value. This can be done because it is not
necessary to perform \posc-\negc splitting to soundly obtain an \inc-\dec
split, as seen in
\cite{chauhan2022efficiently,liu2022abstraction,10.1145/3644387}. Then,
following \cite{cegar-nn}, a sound 
abstraction can be obtained by \textit{merging} all similarly 
labeled neurons as follows: if the neurons have
the label \inc (\dec), replace incoming edges from the same
previous layer neuron with a single edge with the maximum (minimum) of the
original edge weights. Outgoing edges to the same next layer neuron are replaced
with a single edge with the sum of the weights for both the \inc and \dec cases.

\begin{figure}[htbp]
    \centering
    \begin{tikzpicture}[scale=0.5] % Adjust the scale factor as needed
      % Your TikZ code goes here
      \draw[line width=1pt, fill=pink!30] (5,-5) circle (1cm);
      \node at (5,-5) {$n_{(3,1)}$};
      \draw[line width=1pt, fill=pink!30](5,-2) circle (1cm);
      \node at (5,-2) {$n_{(2,1)}$};
      \draw[line width=1pt, fill=blue!30](0,2) circle (1cm);
      \node at (0,2) {$n_{(0,0)}$};
      \node[text=red] at (-2,2) {\textbf{[-1, 1]}};
      \draw[line width=1pt, fill=blue!30](0,-2) circle (1cm);
      \node[text=red] at (-2,-2) {\textbf{[-1, 1]}};
      \node at (0,-2) {$n_{(1,0)}$};
      \draw[line width=1pt, fill=yellow!50](10,0) circle (1cm);
      \node at (10,0) {$n_{(0,2)}$};
      \node[text=red] at (12,0) {$<$\textbf{6.1}};
      \draw[line width=1pt, , fill=pink!30](5, 5) circle (1cm);
      \node at (5,5) {$n_{(0,1)}$};
      \draw[line width=1pt, , fill=pink!30](5, 2) circle (1cm);
      \node at (5,2) {$n_{(1,1)}$};

    \draw[myarrow] (1, -2) to (4, -5);
    \node at (3.25,-4.75) {0.5};
    \draw[ myarrow] (1, 2) to (4, -5);
    \node at (4,-4) {1};
    \draw[ myarrow] (1, -2) to (4, -2);
    \node at (3.45,-2.5) {0.55};
    \draw[ myarrow] (1, 2) to (4, -2);
    \node at (4.25,-0.75) {0.95};
    \draw[ myarrow] (1, -2) to (4, 2);
    \node at (4.25, 0.75) {0.95};
    \draw[ myarrow] (1, -2) to (4, 5);
    \node at (4, 4) {1};
    \draw[ myarrow, solid, postaction={decorate, decoration={text along path,
          text={0.55}, text align=right, raise=1mm}}] (1, 2) to (4, 2);
    \draw[ myarrow, solid, postaction={decorate, decoration={text along path,
          text={0.5}, text align=right, raise=1mm}}] (1, 2) to (4, 5);

    \draw[ ->,>=stealth, solid, postaction={decorate, decoration={text along path,
        text={1}, text align=center, raise=1mm}}] (6,-5) to (9, 0);
    \draw[ myarrow, solid, postaction={decorate, decoration={text along path,
        text={1}, text align=center, raise=1mm}}] (6,-2) to (9, 0);
    \draw[ myarrow, solid, postaction={decorate, decoration={text along path,
        text={1}, text align=center, raise=1mm}}] (6,2) to (9, 0);
    \draw[ myarrow, solid, postaction={decorate, decoration={text along path,
        text={1}, text align=center, raise=1mm}}]  (6, 5) to (9, 0);
      
    \end{tikzpicture}
    \caption{Original Network and Property }
    \label{fig:Original_Net_Property}
  \end{figure}
  

\begin{figure}[htbp]
    \centering
    \begin{tikzpicture}[scale=0.5] % Adjust the scale factor as needed
      % Your TikZ code goes here
      \draw[line width=1pt, fill=orange!30] (5,0) circle (1cm);
      \node at (5,0) {$a_1$};
      \node [text=red] at (5,1.5) {\textbf{1.55}};
      \draw[line width=1pt, fill=blue!30] (0,2) circle (1cm);
      \node at (0,2) {$n_{(0,0)}$};
      \node [text=red] at (-2,2) {\textbf{0.55}};
      \draw[line width=1pt, fill=blue!30] (0,-2) circle (1cm);
      \node [text=red] at (-1.5,-2) {\textbf{1}};
      \node at (0,-2) {$n_{(1,0)}$};
      \draw[line width=1pt, fill=green!30] (10,0) circle (1cm);
      \node at (10,0) {$n_{(0,2)}$};
      \node [text=red] at (12,0) {\textbf{6.2}};
    \draw[myarrow] (1, -2) to (4, 0);
    \node at (2.5,-1.25) {1};
    \draw[myarrow] (1, 2) to (4, 0);
    \node at (2.5,1.25) {1};


    \draw[->,  >=angle 45, thick, postaction={decorate, decoration={text along path,
        text={4}, text align=center, raise=1mm}}] (6,0) -- (9, 0);
    % \draw[solid, postaction={decorate, decoration={text along path,
    %     text={1}, text align=center, raise=1mm}}] (6,-2) -- (9, 0);
    % \draw[solid, postaction={decorate, decoration={text along path,
    %     text={1}, text align=center, raise=1mm}}] (6,2) -- (9, 0);
    % \draw[solid, postaction={decorate, decoration={text along path,
    %     text={1}, text align=center, raise=1mm}}]  (6, 5) -- (9, 0);
      
    \end{tikzpicture}
    \caption{Fully Abstracted Network}
    \label{fig:full_abstract_net}
  \end{figure}
  


For example, consider the network and property in Figure
\ref{fig:Original_Net_Property}. For this example, all the neurons in the output
and middle layer get classified as \inc. Thus, they are all merged together,
leading to the network in Figure \ref{fig:full_abstract_net}.

Note that this process only considers the syntactic structure of the network, no
semantic information is used.

\subsection{ Syntactic Refinement }

The fully merged network obtained in Section \ref{s:nn-sam} may not be
sufficiently strong to be able to dispatch, that is, there may be 
spurious counterexamples. 
In such situations, a common approach to obtaining a better-quality \abs is to
perform refinement steps based on a \gencex $\vct{\beta}$
\cite{cegar-nn,cegarette,cleverest-nn}. 
In existing techniques, this is typically done by restoring a single neuron
coming from \cnc that had been merged with other neurons in \abs. The neuron
chosen is typically one whose contribution to $\vct{\beta}$ is estimated to be
the highest.

These techniques, however, do not consider any semantic behavior to guide their
refinement. As such, the refinement process tends to produce a large number of
restored neurons that are not merged with any other neurons. A proliferation of
these \emph{singleton} neurons lead to \abs having a larger size. At the same
time, a large group of neurons remains merged in \abs, which affects the quality
of \abs. 

We can see this in our example. Say the fully merged network in Figure
\ref{fig:full_abstract_net} we get a $\vct{\beta}$ given by the values in \textbf{bold}.
Then, in the next refinement step, the neuron $\nr{3}{1}$ gets restored, giving
us the network in Figure \ref{fig:refine_step_1}. Again, the $\vct{\beta}$
obtained is shown in \textbf{bold}, and the next refinement step restores $\nr{0}{1}$
leading to Figure \ref{fig:refine_step_2}. We see that in the resultant network,
two semantically dissimilar neurons, $\nr{1}{1}$ and $\nr{2}{1}$, remain merged,
while the merges between the similar pairs of neurons $\nr{3}{1}$, $\nr{2}{1}$
and $\nr{1}{1}$, $\nr{0}{1}$ have been un-done. Indeed, the network in Figure
\ref{fig:refine_step_2} still admits spurious counterexamples, as seen by the
values in \textbf{bold}, and we end up refining all the way to the original
network.

\begin{figure}[htbp]
    \centering
    \begin{tikzpicture}[scale=0.5] % Adjust the scale factor as needed
      % Your TikZ code goes here
      \draw[line width=1pt, fill=orange!30] (5,2) circle (1cm);
      \node at (5,2) {$a_2$};
      \node [text=red] at (5,0.75) {\textbf{1.7}};
      \draw[line width=1pt, fill=pink!30] (5,-2) circle (1cm);
      \node at (5,-2) {$n_{(3,1)}$};
      \node [text=red] at (5,-3.5) {\textbf{1.375}};
      \draw[line width=1pt, fill=blue!30] (0,2) circle (1cm);
      \node at (0,2) {$n_{(0,0)}$};
      \node [text=red] at (-2,2) {1};
      \draw[line width=1pt, fill=blue!30] (0,-2) circle (1cm);
      \node [text=red] at (-1.75,-2) {\textbf{0.75}};
      \node at (0,-2) {$n_{(1,0)}$};
      \draw[line width=1pt, fill=green!30] (10,0) circle (1cm);
      \node at (10,0) {$n_{(0,2)}$};
      \node [text=red] at (12,0) {\textbf{6.475}};
    \draw[->, >= angle 45, thick, postaction={decorate, decoration={text along path,
    text={0.5}, text align=center, raise=1mm}}](1, -2) -- (4, -2);
   
    \draw[myarrow] (1, -2) -- (4, 2);
    \node at (3.75,1.25) {1};
  
    \draw[myarrow] (1, 2) -- (4, -2);
    \node at (3.75,-1.25) {1};
  
    \draw[->, >= angle 45, thick, postaction={decorate, decoration={text along path,
    text={0.95}, text align=center, raise=1mm}}] (1, 2) -- (4, 2);
   


    \draw[->, >= angle 45, thick, postaction={decorate, decoration={text along path,
    text={1}, text align=center, raise=1mm}}] (6,-2) -- (9, 0);
    \draw[->, >= angle 45, thick, postaction={decorate, decoration={text along path,
    text={3}, text align=center, raise=1mm}}] (6, 2) -- (9, 0);

    \end{tikzpicture}
    \caption{Refine Step 1: Culprit Neuron is 3}
    \label{fig:refine_step_1}
  \end{figure}
  

\begin{figure}[htbp]
    \centering
    \begin{tikzpicture}[scale=0.5] % Adjust the scale factor as needed
      % Your TikZ code goes here
      \draw[line width=1pt, fill=blue!30] (0,2) circle (1cm);
      \node at (0,2) {$n_{(0,0)}$};
      \node [text=red] at (-2,2) {\textbf{1}};
      \draw[line width=1pt, fill=blue!30] (0,-2) circle (1cm);
      \node [text=red] at (-2,-2) {\textbf{1}};
      \node at (0,-2) {$n_{(1,0)}$};
      \draw[line width=1pt, fill=pink!30] (5,4) circle (1cm);
      \node at (5,4) {$n_{(0,1)}$};
      \node [text=red] at (5,2.5) {\textbf{1.5}};
      \draw[line width=1pt, fill=pink!30] (5,-4) circle (1cm);
      \node at (5,-4) {$n_{(3,1)}$};
      \node [text=red] at (5,-5.5) {\textbf{1.5}};
      \draw[line width=1pt, fill=orange!30] (5,0) circle (1cm);
      \node at (5,0) {$a_3$};
      \node [text=red] at (5,-1.5) {\textbf{1.9}};
      \draw[line width=1pt, fill=green!30] (10,0) circle (1cm);
      \node at (10,0) {$n_{(0,2)}$};
      \node [text=red] at (12,0) {\textbf{6.8}};
    \draw[->, >= angle 45, thick, postaction={decorate, decoration={text along path, 
    text={0.5}, text align=center, raise=1mm}}](1, 2) -- (4, 4);
    \draw[myarrow](1, -2) -- (4, 4);
    \node at (3.75,3) {1};
    \draw[myarrow](1, 2) -- (4, 0);
    \node at (3.75,1) {0.95};
    \draw[myarrow](1, -2) -- (4, 0);
    \node at (3.75,-1) {0.95};
    \draw[myarrow](1, 2) -- (4, -4);
    \node at (3.75,-3) {1};
    \draw[->, >= angle 45, thick, postaction={decorate, decoration={text along path, 
    text={0.5}, text align=center, raise=1mm}}](1, -2) -- (4, -4);
    \draw[->, >= angle 45, thick, postaction={decorate, decoration={text along path,
    text={1}, text align=center, raise=1mm}}] (6,-4) -- (9, 0);
    \draw[->, >= angle 45, thick, postaction={decorate, decoration={text along path,
    text={2}, text align=center, raise=1mm}}] (6, 0) -- (9, 0);
    %\draw[-> , >= angle 45 , thick, postaction={decorate, decoration={text along path,
    %text={2}, text align=center, thick, raise=1mm}}] (6,0) -- (9, 0);
    \draw[->, >= angle 45, thick, postaction={decorate, decoration={text along path,
    text={1}, text align=center, raise=1mm}}] (6,4) -- (9, 0);
   
    % \draw (1, -2) -- (4, 2);
    % \node at (3.75,1.25) {1};
  
    % \draw (1, 2) -- (4, -2);
    % \node at (3.75,-1.25) {1};
  
    % \draw[solid, postaction={decorate, decoration={text along path,
    % text={0.95}, text align=center, raise=1mm}}] (1, 2) -- (4, 2);
   


    % \draw[solid, postaction={decorate, decoration={text along path,
    % text={1}, text align=center, raise=1mm}}] (6,-2) -- (9, 0);
    % \draw[solid, postaction={decorate, decoration={text along path,
    % text={3}, text align=center, raise=1mm}}] (6, 2) -- (9, 0);

    \end{tikzpicture}
    \caption{Refine Step 2: Culprit Neuron is 0}
    \label{fig:refine_step_2}
  \end{figure}
  

