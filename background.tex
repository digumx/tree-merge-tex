\section{Background}
\subsection{Neural Network}

\dmcmt{There was some general definitions of what an nn is in this section,
removed for now, nor really necessary, maybe can add a notation section if at
all necessary}

\sncmt{Notation is important. Will be required in the upcoming sections.}

We conceptualize neural networks as directed acyclic graphs 
comprising three types of layers: the input layer, intermediate hidden layers, 
and the output layer. Our goal is to focus on the abstracting feed forward networks, 
where neuron values are computed based on preceding layer neuron values. 
Neurons in our network are denoted as $n_{(i,j)}$, where `$i$' signifies the 
neuron number in layer `$j$'. The weight matrix between layers `${j-1}$' and `$j$' 
is denoted as `$W_{{j-1}, j}$', and the bias matrix for layer `$i$' is denoted as 
`$B_{i}$'. The value of the `$i^{th}$' neuron in layer `$j$' for a given input vector 
is represented by `$v_{(i,j)}$', and `$V_{j}$' encompasses all such `$v_{(i,j)}$' values for layer 
`$j$'. The function $o_{(i, j)}$ takes a list of input vectors and outputs a list of
corresponding `$v_{(i,j)}$' for a particular neuron $n_{(i, j)}$. 

The computation of `$V_{j}$' involves applying an ``activation function ($\phi$)" 
to the ``weighted sum":

\[V_{j} = \phi(W_{{j-1}, j} \cdot V_{j-1} + B_{j}).\] 

We confine our activation function to the Rectified Linear Unit (ReLU), 
which can be expressed as $V_{j} = \max(W_{j-1, j} \cdot V_{j-1} + B_{j}, 0)$.

\subsection{ Formal Analysis of Neural Networks }
\label{s:form-an}
\dmcmt{This section used to be titled Neural Network Verification, and explained
    the general idea of neural net verif, and some stuff on prop encoding. We
    are not doing exclusively verif. So, some more general and shorter
    discussion on formal analysis of nn added.}

Several techniques and methods have been studied to improve the reliability and
trustworthiness of \dnn deployed in safety critical settings via formal
analysis. This includes verifying \dnn with respect to a given
safety property \cite{reluplex, cegar-nn, deeppoly, cegarette, cleverest-nn,
conv-abs-gk, deep-abstract, lin-comb-abs-jan}, providing formal explanations of
the behavior of the \dnn \cite{minimal-image-fxai, overview-fxai}, and others.
\dmcmt{Is the and others okay? Ref the backdoor attacks work?} To provide the
formal guarantees behind the analysis performed, all of these techniques rely on
making \textit{neural network queries}. 

These neural network queries are of the form $(P, \mcnc, Q)$, and ask if
for all inputs $x$ to $\mcnc$ for which the formula $P$ holds
the formula $Q$ also holds on the output $\mcnc(x)$. While there are several
tools that can handle such queries, like \marabou and \abcrown, scalability
remains an issue, and so reducing the size of \cnc is desirable.

\subsection{Semantic Compressions and Abstractions with Empirical Guarantees}
\label{s:emp-abs}
\dmcmt{What is a better word than 'weak' here? Does 'empirical' work here?}

Several techniques utilise semantic information, typically extracted via
simulation of the \dnn, to obtain \abs. \dmcmt{Say: to improve scalability? }.
Neural network compression techniques \todo{cite}, produce small \abs, but the
behavior of \abs in connection to \cnc is only characterized empirically.
Similarly, some semantic abstraction techniques \cite{lin-comb-abs-jan} provide
bi-simulation guarantees bounding the difference in the behavior of \abs and
\cnc on a finite set of input points, typically a subset of the training
dataset. Since these techniques characterize the behavior of \abs only on a
finite set of
input points, the trust obtained on the connection between \cnc and \abs is only
of a empirical nature.
Other techniques like \cite{deep-abstract} use bi-simulation to lift
interval bound propagation performed on \abs to get sound bounds on \cnc. While
this does provide a sound proof, interval bounds are typically not strong enough
to prove many interesting and practically relevant neural network queries.
\todo{cite}

\subsection{Concrete Guarantees on Neural Network Abstractions}
\label{s:conc-abs}
\kmcmt{We need to add one is good for analysis and one is good for proofs.
Not in negative light.} \dmcmt{ Even for analysis, we are trying to claim that
our concrete guarantees are better than the empirical ones right? We show a case
of proof, and a case of compression, so things are clear right?}

The notion of providing concrete guarantees on the behavior of \abs relative to
\cnc has been formalized in \cite{cegar-nn}. In particular, guarantees of the
form $\forall x, \mabs(x) \geq \mcnc(x)$ \dmcmt{
Should I format this as definition?} are useful as a general notion of concrete
guarantees in many settings. Two such settings are as follows: 

Firstly, since any general neural network query can be converted to a query of
the form $(P, \mcnc, y < c)$ for some $c$ \cite{cegar-nn, reluplex} \todo{See
\cite{cegar-nn}, they have another citation for this encoding} \dmcmt{It seems
that \cite{cegar-nn}, \cite{reluplex} and we have 3 different encodings, but
generally based on the same idea of implementing bool combs via extra nn
layers}, such a guarantee is
useful for dispatching the query on a smaller network \cite{cegar-nn, cegarette,
cleverest-nn}. This immediately makes an \abs with these guarantees useful for
accelerating several formal analysis techniques (Section \ref{s:form-an}).

Furthermore, such guarantees are also useful for safe compression of \dnn.
Consider the case of medical diagnosis\todo{cite} or aircraft
collisions\cite{acasxu}, where
for safety reasons, a classifier should be biased towards not producing false
negatives. Guarantees such as above can formally ensure that the compressed \abs
never produces any more false negatives than \cnc. \todo{Ref next sections}.
\dmcmt{Should this be here?}

Therefore, in this work we focus on developing a framework\dmcmt{Is this word
here okay?} that produces
abstract networks with this guarantee.

\subsection{Quality of Abstractions and Spurious Points}
\label{s:qual}

A generally useful notion with respect to abstraction is \textit{quality},
which we as the number of \textit{spurious inputs} that witness a difference in
the relevant behavior of \abs and \cnc. In the context of using \abs to
accelerate formal analysis of \dnn, these spurious inputs may simply be
\textit{spurious counterexamples} \cite{cegar-nn, cleverst-nn} to a query
involving \abs that are not counterexamples for the same query involving \cnc.
This notion may be generalized to other uses of abstractions as well. For
instance, for safe compressions, they may be inputs from a dataset which
are falsely classified as positive by \abs. \todo{Ref section}

\subsection{Syntactic Neural Network Splitting and Merging}
\label{s:nn-sam}

\dmcmt{Given that this section is critical to understanding our soundness
    guarantees, and that this is mostly existing work, how much space should be
spent here? Can we trust reader to read \cite{cegar-nn}}

\dmcmt{Start a running example here}

To transform \cnc into \abs so that the soundness guarantees hold, we follow
\cite{cegar-nn} to \textit{split} neurons in \cnc into copies labelled with
labels from \{\inc, \dec\} $\times$ \{\posc, \negc\} so that any
increase (decrease) in the value of a neuron labelled
\inc (\dec) only leads to an increase in output value. Then,
following \cite{cegar-nn}, a sound abstraction can be obtained by
\textit{merging} all similarly labelled neurons as follows: if the neurons have
the label \inc (\dec), replace incoming edges from the same
previous layer neuron with a single edge with the maximum (minimum) of the
original edge weights. Outgoing edges to the same next layer neuron are replaced
with a single edge with the sum of the weights for both the \inc and \dec case.

\dmcmt{Do we need to add a reference for 2-class? It may help shorten section.} 

We make a slight modification to the above: as a first optimization step, we
re-merge the two copies of the \abs neurons that are otherwise the same, but
have \posc and \negc labels respectively. This can always be done without
changing the behavior of the network since these copies will have the same \inc,
\dec labels and the same incoming weights. \dmcmt{Is this clear? Is it better to
write this as a lemma, and then put a proof in appendix?} This optimisation
allows us to discard the \posc and \negc class information, and reduces the size
of the maximally merged network.

Note that this process only considers the syntactic structure of the network, no
semantic information is used.

\subsection{ Syntactic Refinement }
\dmcmt{Jan Kretinsky's abstraction is based on semantic info, refinement is not.
Given this, do we really need to talk about Jan's refinement here? If yes,
how to weave it in?}

The fully merged network obtained in Section \ref{s:nn-sam} may not have
sufficient quality (Section \ref{s:qual}) to be useful. For instance, when doing
formal analysis, there may be too many spurious counterexamples. In such
situations, a common approach to obtain a better quality \abs is to perform
refinement steps based on a spurious counterexample $x$ \cite{cegar-nn,
cegarette, cleverest-nn}. \dmcmt{While \cite{cegar-nn} does it only for spurious cexes,
it can be directly extended to spurious inputs. But, I'm speaking in terms of
sp cex here to avoid confusion. Should I say spurious input instead?} In
existing techniques this is typically done restoring a single neuron in \cnc
that had been merged \abs. The neuron chosen is typically one whose contribution
to $x$ is estimated to be the highest.

These techniques, however, do not consider any semantic behavior to guide their
refinement. As such, the refinement process tends to produce a large number of
restored neurons identical to neurons in \cnc, leading to and \abs of large
size, while retaining a single abstract neuron formed by merging a large group
of neurons in \cnc, affecting the quality of \abs.  \todo{example}
