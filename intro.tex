
\section{Introduction}

\todo{Refer to embedded systems, see Hrishikesh's mail}

Advances in Deep Neural Networks (\dnn) have enabled the scalable solution
of several previously intractable problems like image recognition and natural
language processing. Due to this, \dnn have increasingly assumed a central
role across various domains. These include several safety critical domains like
healthcare \cite{b1}, where they contribute significantly to medical diagnostics
and predictive analysis \cite{b2}, and autonomous vehicles, where DNNs serve as
the backbone for sophisticated perception systems, supporting tasks such as
object recognition and decision-making \cite{b3}. 

However, \dnn are well known to be vulnerable to adversarial
attacks\cite{l-bfgs, fgsm, deep-fool, pgd, ground-truth-adv-attack, cw-attack},
backdoor attacks \cite{backdoor-poisoning} etc.  and, being generally
un-interpretable, can often produce 
unexpected behaviours. Therefore, to build trust on safety critical systems 
that utilize \dnn, it is  critical to understand, interpret 
and validate the possible behaviors of these \dnn via formal analysis
\cite{overview-fxai, minimal-image-fxai, backdoor-verification, nn-lander-verif,
camera-verif-dsouza, generalization-verif}.

A number of techniques have been proposed to build trust on the reliability
of \dnn in safety-critical settings, including verification \cite{reluplex,
deeppoly, crown, beta-crown, cegar-nn}  and formal
explainability \cite{overview-fxai, minimal-image-fxai}. For most of these
techniques, it is necessary to make queries to neural
network verification solvers (such as \cite{reluplex, beta-crown}) to ascertain
the trustworthiness of the \dnn.
 However, since solving these queries is NP-Hard (see appendix of
\cite{reluplex}), and real world \dnn are trained for accuracy and not 
size, formal analysis of \dnn often faces scalability issues.

Structural abstraction based on the \textit{syntax} (the local weights and
biases at each
neuron of the \dnn) form the basis of several techniques attempting to alleviate
this issue \cite{cegar-nn, cegarette, cleverest-nn, conv-abs-gk}. These
techniques work by converting a large \textit{concrete} \dnn \cnc into a smaller
\textit{abstract} DNN \abs via \textit{merging} groups of neurons in \cnc into
single neurons in \abs. Each such
merge is done in a way that ensures that there are \textit{concrete}, formal
soundness guarantees linking the behavior of \cnc and \abs. Then, one can make 
queries on \abs, and using the concrete soundness guarantees, lift the result to 
\cnc and argue about its reliability. However, these techniques
do not take into account the global \textit{semantic} behavior of the network,
thus producing potentially sub-optimal abstractions.

On the other hand, neural network compression techniques \cite{dnn-compression}
and semantic abstraction techniques \cite{deep-abstract, lin-comb-abs-jan} take
into account the global \textit{semantic} information within the network.
However,
compression techniques provide no soundness guarantees, while the semantic
abstraction techniques provide limited soundness guarantees. In particular, the
guarantees provided by \cite{deep-abstract} are limited to lifting specific
bound propagation based proofs, and those provided by \cite{lin-comb-abs-jan}
only characterise soundness on a finite subset of the input space. This limits
the applicability of these techniques to situations where concrete, hard
guarantees may be necessary.

In this work we combine the syntactic and semantic approaches into a single
framework for generating an abstract network. By splitting and labelling the
neurons \textit{inc} and \textit{dec} similar to \cite{cegar-nn}, and
restricting ourselves to merges involving only similarly labelled neurons, we
provide concrete soundness guarantees linking the behavior of \cnc and \abs via
syntactic constraints.
On the other hand, to take into account the global semantic behavior we
introduce a semantic closeness metric between two neurons in the same layer.
Using this metric, we construct a tree of merge operations that 
captures the relative contribution of each merge to the quality of the
abstraction. We propose a refinement procedure that uses this
tree as a guide to undo the merge operations that
contribute most to the poor quality of \abs.  We show that in the resulting
refined network, groups of neurons that remain merged are semantically closer
than neurons that get un-merged. Thus, we are able to refine \abs in a way that
is optimal with respect to the semantic behavior.

We assemble these pieces into an abstraction-refinement framework that can
generate high quality \abs by starting with a fully merged network and
iteratively refining until a strong enough network is obtained. We demonstrate
the usefulness of this framework by using it to find safe compressions of \mnist
networks that are guaranteed to not introduce any false negatives, while
minimising the number of false positives introduced. Apart from this, we also
demonstrate the quality of the abstract networks produced for \mnist networks
and $\epsilon$-robustness properties, and evaluate the verification of the
\acasxu benchmarks via a \cegar loop based on our abstraction-refinement
framework.
