\begin{abstract}
Neural Networks are increasingly being used in several safety critical
applications, including medical imaging, self driving cars and aircraft
collision detection. To gain trust on the networks being deployed in safety
critical domains and reduce the risk of safety violations, several techniques
based on formal analysis has been proposed, including formal verification and
formal explainability. However, the scalability of these techniques is highly
sensitive to the size of the networks involved, limiting the applicability of
these techniques to real-world networks. To handle these scalability issues,
structural abstraction based on syntactic techniques have been proposed, and
while these provide formal soundness guarantees, they do not take into account
the semantic behavior of the network. On the other hand, neural network
compression and semantic abstraction \dmcmt{Jan, few lines in intro}
techniques take into account this information, but the soundness guarantees they
provide are generally much weaker, and empirical. We propose to combine both
the semantic and syntactic approaches into a single framework to try to obtain
the best of both worlds, guiding our abstraction using global semantic
information while still providing concrete soundness guarantees based on
syntactic constraints. We do this by constructing a partial order of best
possible merges using global semantic information, and represent it using a
tree-like data structure. Then we follow this tree to perform syntactic merge
and example guided \sncmt{Some other wording?} refinement operations. This
allows us to obtain abstract networks that is guided by global semantic
information while providing concrete soundness guarantees. We demonstrate the
effectiveness of these abstractions on standard neural network benchmarks via
experiments.
\end{abstract}
