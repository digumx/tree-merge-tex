
\begin{abstract}
Deep Neural Networks (DNNs), today, are being trained and trusted for
performing fairly complex tasks, even in safety-critical applications such as
automated driving, medical diagnosis, and air-traffic control. This does not
directly benefit the embedded domain though, primarily because of the resource
constraints in an embedded system. Real-world applications tend to rely on very
large DNNs to achieve the desired accuracy, making it a challenge for them to
be executed on embedded architectures due to power and memory limitations. The
size of these networks is also a bottleneck in proving their trustworthiness
through formal verification or explanation. While there exist several methods to
compress DNNs, there are no formal guarantees linking the behavior of the
compressed and original networks. To address this issue, several
abstraction methods have been proposed. This includes techniques that perform
an abstract analysis of the network, and those that abstract or compress the
network itself.  We study the latter, in which there are two different
approaches that have been taken: \emph{syntactic} abstraction, with formal
soundness guarantees, and \emph{semantic} abstraction where the soundness
guarantees are usually much weaker, and empirical. In this paper, we propose to
combine the semantic and syntactic approaches into a single framework, to get
the best of both worlds.  This allows us to guide the abstraction using global
semantic information while still providing concrete soundness guarantees based
on syntactic constraints.  Our experiments on standard neural network
benchmarks shows that this can be effective for both compression as well as
verification.
\end{abstract}

